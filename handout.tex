\documentclass[a3, ruledsections, 8pt]{sciposter}
\usepackage{lipsum}
\usepackage{epsfig}
\usepackage{amsmath}
\usepackage{amssymb}
\usepackage{multicol}
\usepackage{graphicx,url}
\usepackage[german, english]{babel}
\usepackage[utf8]{inputenc}
\renewcommand{\titlesize}{\LARGE}
\renewcommand{\authorsize}{\small}
\renewcommand{\instsize}{\footnotesize}

%\usepackage{fancybullets}
\newtheorem{Def}{Definition}
\definecolor{SectionCol}{rgb}{1.0,0.6,0.0} %orange


\title{W4 CAA Scripting Languages Hackathon I – Can you code this?}
%Título do projeto

\author{Clemens Schmid$^{1}$, Martin Hinz$^{2}$, Carolin Tietze$^{3}$}
%nome dos autores

\institute
{$^{1}$ Römisch-Germanisches Zentralmuseum Leibniz-Forschungsinstitut für Archäologie: Mainz, Rheinland-Pfalz\\
$^{2}$ Institut für Archäologisches Wissenschaften, Universität Bern \\
$^{3}$ Institut für Klassische Altertumskunde, Christian-Albrechts-Universität zu Kiel
}
%Nome e endereço da Instituição

\email{clemens@nevrome.de, martin.hinz@iaw.unibe.ch, ctietze1991@gmail.com}
% Onde você coloca os emails dos integrantes


%\date is unused by the current \maketitle

\rightlogo[1]{images/front_20180624.png}
\leftlogo[1]{images/49449374.png}
% Exibe os logos (direita e esquerda)
% Procure usar arquivos png ou jpg, e de preferencia mantenha na mesma pasta do .tex
%%%%%%%%%%%%%%%%%%%%%%%%%%%%%%%%%%%%%%%%%%%%%%%%%%%%%%%%%%%%%%%%%%%%%%%%%%%%%%%%
%%% Begin of Document



\begin{document}
%define conference poster is presented at (appears as footer)

\conference{{\bf CAA 2019} Check Object Integrity - 47th Computer Applications and Quantitative Methods in Archaeology - 23-27 April 2019}

%\LEFTSIDEfootlogo
% Uncomment to put footer logo on left side, and
% conference name on right side of footer

% Some examples of caption control (remove % to check result)

%\renewcommand{\algorithmname}{Algoritme} % for Dutch

%\renewcommand{\mastercapstartstyle}[1]{\textit{\textbf{#1}}}
%\renewcommand{\algcapstartstyle}[1]{\textsc{\textbf{#1}}}
%\renewcommand{\algcapbodystyle}{\bfseries}
%\renewcommand{\thealgorithm}{\Roman{algorithm}}

\maketitle

%%% Begin of Multicols-Enviroment
%\begin{multicols}{1}

%%% Abstract
%\begin{abstract}
%No modelo que a Danielle postou não tinha resumo, então confirmem para saber se precisa ou não.
%Morphological pattern spectra computed from granulometries are frequently used
%to classify the size classes of details in textures and images. An extension
%of this technique, which retains information on the spatial
%distribution of the details in each size class is developed. Algorithms for
%computation of these spatial pattern spectra for a large number of
%granulometries on binary images are presented.
%\end{abstract}

%%% Introduction

\section{General Information}

\begin{itemize}
\item A \textbf{repository} with all information and data for this workshop is available at \newline \url{https://github.com/sslarch/caa2019_hackathon}
\item 2-4 \textbf {groups} are formed on site according to framework preference and skill levels (\textit{Unconference style}).
\item You have \textbf{3} hours to work on all tasks, breaks can be taken as you wish. It's not necessary to complete all possible tasks. Work as far as you can go. This is \textbf{NOT} an exam.
\item All results must be submitted in one \textbf{reproducible report} with all code and plots. This can be rendered from IPython Notebook, Rmarkdown, Latex, etc. or compiled manually. Ideally the report is submitted as a Pull Request to the hackathon repository on github. The file(s) should be added to the \verb|reports| directory.
\item The organizers of this workshop are available for questions and advice. They are able to assist you with problems as far as they are familiar with your toolset.
\end{itemize}

\section{Dataset}

The data for this excercise --- \verb|Michelsberg| --- are taken from the R package \verb|archdata| (Carlson/Roth 2018). It's a features by types table of abundance data on vessel types in archaeological features of the Younger Neolithic Michelsberg Culture from Belgium, France and Germany by Birgit Höhn (2002). The 109 observations/lines represent individual features (pits, ditches, etc.) from sites. For each feature we have information about 42 variables/columns. These include identifiers, classified pottery type counts, phase attribution and site coordinates.

\bigskip

\textit{From ?archdata::Michelsberg:}

\begin{itemize}
\item id: Unique identifier (categorical, integer)
\item site\_name: Name of site (categorical, character)
\item catalogue\_nr: Number in catalogue of Höhn (2002) (categorical, integer)
\item feature\_nr: Number of the archaeological feature (categorical, numeric)
\item to3: Pot/vessel type 3 count
\item f4: Bottle type 4 count
\item b2: Beaker type 2 count
\item to2: Pot/vessel type 2 count
\item b3: Beaker type 3 count
\item b7: Beaker type 7 count
\item kw5: Carinated bowl type 5 count
\item vg1: Storage vessel type 1 count
\item vg2: Storage vessel type 2 count
\item t4a: Tulip beaker type 4a count
\item kw2: Carinated bowl type 2 count
\item kw4: Carinated bowl type 4 count
\item b5: Beaker type 5 count
\item t3b: Tulip beaker type 3b count
\item f3: Bottle type 3 count
\item kw3: Carinated bowl type 3 count
\item kw1: Carinated bowl type 1 count
\item b6: Beaker type 6 count
\item to1: Pot/vessel type 1 count
\item b1: Beaker type 1 count
\item t3a: Tulip beaker type 3a count
\item vg4: Storage vessel type 4 count
\item ks2: Conical bowl type 2 count
\item ks1: Conical bowl type 1 count
\item t2b: Tulip beaker type 2b count
\item f2: Bottle type 2 count
\item bs3: Globular bowl type 3 count
\item t2a: Tulip beaker type 2a count
\item bs2: Globular bowl type 2 count
\item b4: Beaker type 4 count
\item bs1: Globular bowl type 1 count
\item f1: Bottle type 1 count
\item t1b: Tulip beaker type 1b count
\item vg3: Storage vessel type 3 count
\item t1a: Tulip beaker type 1a count
\item mbk\_phase: MBK phase according to Lüning (1967) as an ordered factor with levels $<$ I/II $<$ II $<$ II/III $<$ III $<$ III-V $<$ III/IV $<$ IV $<$ IV/V $<$ Munz $<$ V
\item x\_utm32n: x coordinate in m; projection UTM WGS 84, zone 32 nord
\item y\_utm32n: y coordinate in m; projection UTM WGS 84, zone 32 nord
\end{itemize}

\medskip

Höhn (2002) recorded pottery vessel shapes from 108 archaeological features (pits, ditches etc.) from 69 sites of the Central European Younger Neolithic Michelsberg Culture (MBK; 4350-3500 BC) following Lüning's (1967) typology. Her correspondence analysis of the abundance data (columns 5 to 39) exhibits a pronounced Guttman effect or arch, suggesting the data set is structured by a time gradient. Recently Mischka et al. (2015) projected an 109th Michelsberg assemblage, Flintbek LA48, a pit with Michelsberg pottery from a North German site of the Funnel Beaker Culture (TRB), as a supplementary row into the existing chronology thereby connecting the relative chronologies of TRB and MBK. The data frame contains as attributes the references for the data, a typological key and the map projection. Note that ambiguous fragments of conical bowls (ks1 and ks2) are assigned as 0.5 to each of the two types resulting also in positive entries suitable to analysis by CA.

\medskip

Höhn, B. 2002. Die Michelsberger Kultur in der Wetterau. Universitätsforschungen zur prähistorischen Archäologie 87. Bonn: Habelt.

Mischka, D., Roth, G. and K. Struckmeyer 2015. Michelsberg and Oxie in contact next to the Baltic Sea. In: Neolithic Diversities. Perspectives from a conference in Lund, Sweden. Acta Archaeologica Lundensia Ser. 8, No. 65, edited by Kr. Brink et al., pp 241–250.

Lüning, J. 1967. Die Michelsberger Kultur: Ihre Funde in zeitlicher und räumlicher Gliederung. Berichte der Römisch-Germanischen Kommission 48, 1-350.

\pagebreak

\section{Tasks}

\begin{enumerate}
\item Create a two-columns table with information about the sites (\verb|site_name|) and the amount of features (\verb|feature_name|) documented per sites. Who many sites are represented in the table by more than one feature? 
\item Create a two-columns table with information about the material  (\verb|to3|, \verb|f4|, ..., \verb|t1a|)
\item Grouped counts of material by \verb|site_name| \& \verb|mbk_phases| and further cross tables
\item Visualisation of grouped counts in plot matrizes. For \verb|mbk_phases| these can be constructed as time series plots
\item Spatial map of sites with mapping of counts computed in task 3.
\item Correspondence Analysis (CA) of material variables
\item 2D and 3D Visualisation of CA results with mapping of \verb|site_name| \& \verb|mbk_phases|
\item Mapping of CA axis rank on spatial map
\item \textbf{Bonus} Chi-square distance between all material variables and network visualisation and analysis
\end{enumerate}

%%% References

%% Note: use of BibTeX als works!!

%\bibliographystyle{plain}
%\begin{thebibliography}{1}
%
%\bibitem{Flusser:Suk:93}
%J.~Flusser and T.~Suk.
%\newblock Pattern recognition by affine moment invariants.
%\newblock {\em Pattern Recognition}, 26:167--174, 1993.
%
%\bibitem{Hu:62}
%M.~K. Hu.
%\newblock Visual pattern recognition by moment invariants.
%\newblock {\em IRE Transactions on Information Theory}, IT-8:179--187, 1962.
%
%\bibitem{maragos89:_patter}
%P.~Maragos.
%\newblock Pattern spectrum and multiscale shape representation.
%\newblock {\em IEEE Trans. Patt. Anal. Mach. Intell.}, 11:701--715, 1989.
%
%\bibitem{Meijster:Wilkinson:PAMI}
%A.~Meijster and M.~H.~F. Wilkinson.
%\newblock A comparison of algorithms for connected set openings and closings.
%\newblock {\em IEEE Trans. Patt. Anal. Mach. Intell.}, 24(4):484--494, 2002.
%
%\bibitem{Nacken:thesis}
%P.~F.~M. Nacken.
%\newblock {\em Image Analysis Methods Based on Hierarchies of Graphs and
%  Multi-Scale Mathematical Morphology}.
%\newblock PhD thesis, University of Amsterdam, Amsterdam, The Netherlands,
%  1994.
%
%\end{thebibliography}

%\end{multicols}

\end{document}
